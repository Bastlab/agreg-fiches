\documentclass{agregfiche}

\title{Leçon 915 -- Classes de complexité. Exemples}

\begin{document}

\maketitle

\secrapports

\begin{rapport}{2017}

Le jury attend que le candidat aborde à la fois la complexité en temps et en
espace. Il faut naturellement exhiber des exemples de problèmes appartenant aux
classes de complexité introduites, et montrer les relations d’inclusion
existantes entre ces classes, en abordant le caractère strict ou non de ces
inclusions. Le jury s’attend à ce que les notions de réduction polynomiale, de
problème complet pour une classe, de robustesse d’une classe vis à vis des
modèles de calcul soient abordées. Parler de décidabilité dans cette leçon
serait hors sujet.

\end{rapport}

\secindispensables

\begin{itemize}
    \item La notion de complexité en espace et en temps
        avec les inclusions évidentes entre elles
    \item Le problème "classiquement complet"
        \begin{equation}
            L = \left\{ (M, x, 1^t) ~|~ \langle M, x \rangle \text{ termine en
                temps (resp. espace) }\leq t
            \right\}
        \end{equation}
    \item Les classes usuelles ($\P$, $\NP$, $\PSPACE$)
    \item Notion de réduction, de problème complet

\end{itemize}

\secasavoir

\begin{itemize}

	\item La robustesse vis-à-vis du modèle de calcul 
	(déterministe, non-déterministe, une bande, plusieurs 
	bandes, écriture sur entrée)
	\item Savitch et les classes en espace
	\item Exemples de problèmes complets sur des 
	domaines différents (graphes, automates,
	grammaires, logique, circuits)
	\item Les théorèmes de hiérarchie stricte
	\item Un petit diagramme qui permet de s'y retrouver 
	en annexe avec les inclusions strictes et les égalités
\end{itemize}


\secidees

\begin{itemize}
    \item Les théorèmes de simulation pour les machines 
        universelles avec des bornes fines de complexité
    \item Les classes alternantes, le lien avec 
        EXPSACE et la hiérarchie polynômiale (Carton)
    \item Les classes $NL$, $co-NL$ et Immerman-Szelepcsényi
    \item Algorithmes randomisés comme autre modèle de calcul
        (tests de primalité, algorithme de Berlekamp,
        polynomial identity testing, RP, coRP)
    \item Des théorèmes de borne inférieure de simulation
        (une bande vs deux bandes en $O(n)$ vs $O(n^2)$ 
        pour le langage des palindromes).
    \item Une machine qui calcule trop rapidement est un automate 
        (Carton)
    \item Développer le côté logique (2SAT, HORNSAT, SAT, QBF, CTLSAT).
    \item Les arguments de padding peuvent faire de jolis résultats 
        ($NSPACE = SPACE$ via Savitch, si $P = NP$ alors $EXP = NEXP$)
    \item S'il existe un problème $NP$-complet sur un langage unaire,
        alors $P = NP$.
\end{itemize}

\secpieges

\begin{itemize}
    \item Ne pas centrer toute la leçon sur $P$ et $NP$
    \item Bien expliquer la notion de réduciton polynômiale,
        et logarithmique si on parle de $NL$
    \item Si on parle de $NL$, bien formaliser le type de machines 
        considérées
    \item Subtilités avec les classes complémentaires, problèmes à promesses
\end{itemize}


\secquestionsclassiques

\begin{itemize}
    \item Quels sont les impacts de la variation du modèle sur la complexité ?
    \item Pourquoi utiliser le formalisme des machines de Turing 
        plutôt qu'un autre ($\lambda$-calcul, fonctions récursives) ?
    \item Citer une classe qui n'est pas $P$, $NP$ ou $PSPACE$
    \item Montrer que tel problème $X$ est $C$-complet.
    \item Comment varie la complexité en fonction de la taille de l'alphabet
\end{itemize}

\secreferences

\begin{itemize}
\item \temporary{Carton}
\item \temporary{Arora Barak}
\end{itemize}

\secdev

\begin{itemize}
    \item \temporary{Hiérarchie en temps et en espace}
    \item \temporary{Immerman Szelepcseniy}
    \item \temporary{NP-complétude d'un problème}
    \item \temporary{2SAT est $NL$-complet et en temps linéaire sur une 
        machine RAM}
    \item \temporary{Universalité d'un langage rationnel est PSPACE complet}
\end{itemize}


\end{document}
