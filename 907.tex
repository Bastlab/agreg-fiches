\documentclass{agregfiche}

\title{Leçon 907 -- Algorithmique du texte. Exemples et applications.}

\begin{document}

\maketitle

\secrapports

\begin{rapport}{2017}

    Cette leçon devrait permettre au candidat de présenter une grande variété d’algorithmes et de paradigmes de programmation, et ne devrait pas se limiter au seul problème de la recherche d’un motif dans un texte, surtout si le candidat ne sait présenter que la méthode naïve. De même, des structures de données plus riches que les tableaux de caractères peuvent montrer leur utilité dans certains algorithmes, qu’il s’agisse d’automates ou d’arbres par exemple. Cependant, cette leçon ne doit pas être confondue avec la 909, «Langages rationnels et Automates finis. Exemples et applications.». La compression de texte peut faire partie de cette leçon si les algorithmes présentés contiennent effectivement des opérations comme les comparaisons de chaînes : la compression LZW, par exemple, est plus pertinente dans cette leçon que la compression de Huffman.

\end{rapport}

\secindispensables

\begin{itemize}
    \item Les algorithmes naïfs de recherche de motif
    \item L'approche par automate au problème
    \item Des exemples qui ne sont pas seulement de la recherche 
        de motif ! (PLSC, distance d'édition...)
    \item Les algorithmes non naïfs de KMP et Boyer-Moore.
\end{itemize}

\secpieges

\begin{itemize}
    \item Bien préciser quels sont les problèmes qu'on cherche 
        à résoudre, les entrées, et les complexités des algorithmes.
        Par exemple, faire la différence entre la recherche 
        d'un motif, la recherche de $k$ motifs, la recherche 
        d'une expression régulière \dots

    \item Pour la complexité, ne pas oublier de 
        considérer la taille de l'alphabet comme un paramètre !

    \item Il faut faire des dessins en algorithmique du texte, sans 
        quoi on tombe dans des formulations peu claires (et souvent 
        fausses) et des preuves incompréhensibles.

    \item Faire une simple liste d'algorithme de texte n'est pas 
        une bonne idée, il faut les organiser de manière pédagogique.
        Cela peut vouloir dire par "problème", par "méthode", ou 
        bien par "complexité".

    \item Surtout bien préciser au début de la leçon quel est le 
        domaine étudié de manière précise. Le domaine est trop vaste 
        pour tenir en une seule leçon: l'analyse lexicale et 
        syntaxique par exemple doit être évitée, bien que techniquement
        relevant de l'algorithmique du texte.

    \item Éviter de faire un développement trop technique au tableau 
        car ce n'est agréable pour personne. Il faut rechercher 
        des arguments intuitifs plus qu'élémntaires afin de rendre 
        plus digestes les preuves.

    \item On ne peut pas faire l'impasse sur les fonctions bordures,
        et plus généralement les notions élémentaires de l'algorithmique
        du texte (préfixe, suffixe, table des périodes etc).
\end{itemize}

\secidees

On peut penser à séparer les algorithmes \emph{calculatoires}
de type KMP, Boyer-Moore etc. et les \emph{constructions}
par exemple les automates, les arbres à suffixes. La programmation
dynamique se situant un peu entre les deux.

L'algorithme de Karp-Rabin fait usage d'une structure intéressante
de \emph{hash cyclique} qui peut faire l'objet d'une remarque.

Il peut être intéressant de parler de l'automate de Simon et 
sa complexité amortie, ainsi que les compromis temps/mémoire
qu'il apporte. On peut d'ailleurs voir l'algorithme de Simon
et l'automate dans deux parties séparées pour avoir plusieurs 
angles d'attaque sur cet objet.

Parler des structures (automates, Aho-Corasick, Simon,
Suffix Tree, Suffix Trie, Suffix Array) permet d'avoir 
des preuves plus agréables syntaxiquement et de remplir 
une partie de la leçon.

Pleins d'algorithmes de programmation dynamique peuvent 
être utilisés dans cette leçon. On pensera en particulier 
à la recherche de motif \emph{approximative} qui est un 
très joli résultat.

On peut aussi parler de compression de texte et codes 
correcteurs, mais il faut
alors savoir dire des choses non triviales tout en restant 
proche de "l'esprit algorithmique". Parler par exemple 
d'entropie ou de bornes théoriques semble clairement 
en dehors de la leçon.

\secquestionsclassiques

\begin{itemize}
    \item TODO
\end{itemize}

\secreferences

\begin{itemize}
\item \temporary{Crochemore}
\item \temporary{Jewels of Stringology}
\end{itemize}

\secdev

\begin{itemize}
    \item \temporary{CYK}
    \item \temporary{KMP}
    \item \temporary{Aho-Corasick}
    \item \temporary{Distance d'édition}
\end{itemize}


\end{document}
