\documentclass{agregfiche}

\title{Leçon 907 -- Algorithmique du texte. Exemples et applications.}

\begin{document}

\maketitle

\secrapports

\begin{rapport}{2017}

    Cette leçon devrait permettre au candidat de présenter une grande variété d’algorithmes et de paradigmes de programmation, et ne devrait pas se limiter au seul problème de la recherche d’un motif dans un texte, surtout si le candidat ne sait présenter que la méthode naïve. De même, des structures de données plus riches que les tableaux de caractères peuvent montrer leur utilité dans certains algorithmes, qu’il s’agisse d’automates ou d’arbres par exemple. Cependant, cette leçon ne doit pas être confondue avec la 909, «Langages rationnels et Automates finis. Exemples et applications.». La compression de texte peut faire partie de cette leçon si les algorithmes présentés contiennent effectivement des opérations comme les comparaisons de chaînes : la compression LZW, par exemple, est plus pertinente dans cette leçon que la compression de Huffman.

\end{rapport}

\secindispensables

\begin{itemize}
    \item Les algorithmes naïfs de recherche de motif et la complexité
    \item Les algorithmes non naïfs de KMP et Boyer-Moore.
    \item Recherche de motif par automate
    \item Des exemples qui ne sont pas seulement de la recherche 
        de motif ! (PLSC, distance d'édition)
    \item On ne peut pas faire l'impasse sur les fonctions bordures,
        et plus généralement les notions élémentaires de l'algorithmique
        du texte (préfixe, suffixe, table des périodes)
    \item Penser à justifier que les pire des cas sont atteints pour certains
        mots
\end{itemize}

\secpieges

\begin{itemize}
    \item Bien préciser quels sont les problèmes qu'on cherche 
        à résoudre, les entrées, et les complexités des algorithmes.
        Par exemple, faire la différence entre la recherche 
        d'un motif, la recherche de $k$ motifs, la recherche 
        d'une expression régulière 

    \item Pour la complexité, ne pas oublier de 
        considérer la taille de l'alphabet comme un paramètre !

    \item Il faut faire des dessins en algorithmique du texte, sans 
        quoi on tombe dans des formulations peu claires (et souvent 
        fausses) et des preuves incompréhensibles.

    \item Faire une simple liste d'algorithme de texte n'est pas 
        une bonne idée, il faut les organiser de manière pédagogique.
        Cela peut vouloir dire par "problème", par "méthode", ou 
        bien par "complexité".

    \item L'analyse lexicale et syntaxique par exemple doit être évitée, bien que techniquement
        relevant de l'algorithmique du texte.

\end{itemize}

\secidees

\begin{itemize}
    \item L'algorithme de Karp-Rabin
    \item Automate de Simon
    \item Regarder KMP sous l'angle des structure de données 
        ou de l'algorithme (automate de Simon)
    \item 
        On peut aussi parler de compression de texte et codes 
        correcteurs, mais bien attention à rester dans la leçon.
    \item 
        Parler des structures (automates, Aho-Corasick, Simon,
        Suffix Tree, Suffix Trie, Suffix Array, Prefix Trie).

    \item Recherche de motif approximative
\end{itemize}

\secquestionsclassiques

\begin{itemize}
    \item Quel algorithme est utilisé par Grep ? Libre office ?
    \item Faire la table de bordure de tel motif.
    \item Calculer l'automate des occurences et le faire tourner sur un exemple.
    \item Donner un exemple de pire des cas pour les différents algorithmes.
    \item Pourquoi est-ce que l'automate de Simon possède un nombre linéaire
        d'arc retours ?
\end{itemize}

\secreferences

\begin{itemize}
\item \temporary{Crochemore}
\item \temporary{Jewels of Stringology}
\end{itemize}

\secdev

\begin{itemize}
    \item \temporary{CYK}
    \item \temporary{KMP}
    \item \temporary{Aho-Corasick}
    \item \temporary{Distance d'édition}
\end{itemize}


\end{document}
