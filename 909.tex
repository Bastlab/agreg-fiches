\documentclass{agregfiche}

\title{Leçon 909 -- Langages rationnels et automates finis.  Exemples et applications.}

\begin{document}

\maketitle

\secrapports

\begin{rapport}{2018}
Pour cette leçon très classique, il importe de ne pas oublier de donner exemples et applications, ainsi
que le demande l’intitulé.
Une approche algorithmique doit être privilégiée dans la présentation des résultats classiques (déterminisation, théorème de
Kleene, etc.) qui pourra utilement être illustrée par des exemples. Le jury pourra naturellement poser des questions telles que : connaissez-vous un algorithme pour décider de l’égalité des langages reconnus par deux automates ? quelle est sa complexité ?
Des applications dans le domaine de l’analyse lexicale et de la compilation entrent naturellement dans
le cadre de cette leçon.
\end{rapport}

\secindispensables

\begin{itemize}
\item  Définitions des automates finis et de la notion de langages reconnaissables. Exemples de languages reconnaissables.
\item Existence de langages non reconnaissables. Pumping Lemma. Exemples
\item Applications, par exemple~: la compilation, la recherche de motif, le
    model checking (et autres).
\item Accessibilité, Complétude, déterminisation, minimisation (Nerode).
\item Propriétés de clôture des langages reconnaissables, et construction
    effective sur les automates.
\item Ne pas oublier les complexités des algorithmes.
\item Définitions des expressions rationnelles et langages rationnels.
\item Théorème de Kleene, connaître/mentionner les constructions pour la preuve (parmis Thompson, Glushkov, Antimirov, McNaughton-Yamada, Brozoswky).
\end{itemize}


\secpieges

\begin{itemize}
\item Si l'analyse lexical et de la compilation peuvent être un exemple d'application, ils ne doivent pas prendre une place prépondérante. Les automates à piles ne sont notamment pas au programme. On ne rentrera pas dans de nombreuses définitions pour les grammaires.
\item Cette leçon est basique et fondamentale, il faut très bien couvrir toutes les bases, et éviter de partir trop loin théoriquement.
\end{itemize}

\secidees
% to complete
\begin{itemize}
\item Analyse lexicale et chaîne de compilation.
\item Des questions de complexités
\item Divers algorithmes de minisation (Hopcroft, Brozoswky)
\item Reconnaissance par monoïdes.
\item Lien avec les autres modèles de calcul (Boustrophédon, Machine de turing
    qui n'écrivent pas sur leur entrée).
\item Lien avec la logique (MSO)
\end{itemize}



\secquestionsclassiques
\begin{itemize}
\item Comment tester le vide d'un automate ? Commenter tester l'acceptation d'un mot ? L'inclusion de langages ? L'égalité de langages ? Obtenir le complémentaire d'un language ? Complexités ? (attention aux inputs, regexp vs. $\mathcal{A}$ déterministe vs. $\mathcal{A}$ non déterministe)
\item Dur : Tester l'universalité d'une expression rationnel ?
\item Automate dont le determinisé est en $2^{|Q|}$ ?
\item Quelle classe de complexité est reconnu par les automates ?
\item Comment prouver la minimalité d'un automate ?
\item Prouver que tel langage $X$ n'est pas régulier.
\end{itemize}

\secreferences
\begin{itemize}
\item \reference{Car}{Langages formels, calculabilité et complexité}{\bsc{Carton}}{à la BU/LSV}{Très bonne référence couvrant beaucoup de bases. Se méfier de certaines preuves faites un peu rapidement.}  

\item \temporary{Beauquier}
\item \temporary{Sakarovitch}
\end{itemize}

\secdev
%% sketchy, to complete
\begin{itemize}
\item \dev{Théorème de Klenne}{[Car]}{Thm 1.59 p.36}{907,909,923}{Tout faire est ambitieux, mais cela passe si on prend les constructions les plus basiques (Thomson et McNaughton-Yamada). Si on fait par contre Antimirov, cela peut suffire.}

\item \dev{Décidabilité de l'arithmétique de Presburger}{[Car]}{Thm 3.63 p.164}{909,914,924}{Idée générale simple, mais attention aux détails.}

\item \dev{Automate des occurences}{[Cor]}{p.886}{909}{Tiens bien en 15 min, mais attention à bien maitriser les petits calculs. Extension possible vers KMP.}

\item \temporary{Aho Corasick (Crochemore)}
\item \temporary{Boustrophédon Carton}
\item \temporary{Nerode}

\end{itemize}


\end{document}
