\documentclass{agregfiche}

\title{Leçon 913 -- Machines de Turings. Applications.}

\begin{document}

\maketitle

\secrapports

\begin{rapport}{2018}

Il s'agit de présenter un modèle de calcul.
Le candidat doit expliquer l'intérêt de disposer d'un modèle formel de calcul et discuter le choix des machines de Turing.
La leçon ne peut se réduire à la leçon 914 ou à la leçon 915, même si, bien sûr, la complexité et l'indécidabilité sont des exemples d'applications.
Plusieurs développements peuvent être communs avec une des leçons 914, 915, mais il est apprécié qu'un développement spécifique soit proposé, comme le lien avec d'autres modèles de calcul, ou le lien entre diverses variantes des machines de Turing.

\end{rapport}

\secindispensables

\begin{itemize}
    \item Définitions des machines de Turing déterministes, non déterministes.
    \item Reconnaissabilité (mot, langage).
    \item Machine de Turing universelle.
    \item Décidabilité : définition d'un problème décidable, indécidable, réductions.
    \item Indécidabilité de l'arrêt.
    \item Calculabilité : définition d'une fonction calculable.
    \item Exemples de fonctions calculables : fonctions arithmétiques.
    \item Expressivité : équivalence avec les fonctions récursives, le lambda-calcul.
    \item Complexité : définition de la complexité en temps, espace.
\end{itemize}

\secpieges

\begin{itemize}
    \item Cf rapport, avoir une leçon trop inspirée des leçons 914, 915.
    \item Chaque livre a une définition des machines de Turing : il faut rester cohérent, et faire attention aux résultats utilisés.
    \item Bien justifier le formalisme introduit et son utilité en comparaison
        aux autres modèles
    \item Bien faire la différence entre la reconnaissabilité et la
        calculabilité, ainsi que le lien entre les deux.
\end{itemize}

\secidees

\begin{itemize}
    \item Stabilité des notions pour les variantes des machines de Turing : non
        déterminisme (existentiel, universel, alternant), 
        plusieurs rubans, alphabet ou nombre d'états restreints.
    \item Ouvertures relevant des leçons 914 et 915.
    \item Définition des classes de complexité, réductions polynomiales et logarithmiques.
    \item Limites de la stabilité (machines n'écrivant pas sur leur entrée,
        calculant en temps $o(\log \log n)$)
    \item Lien avec des notions plus faibles de calcul (automates, grammaires
        algébriques)
    \item Lien avec les machines de Minsky
    \item Faire le lien entre les variantes de transition (non-déterminisme,
        probabilités) et l'ajout d'une bande d'advice
\end{itemize}

\secquestionsclassiques

\begin{itemize}
    \item Quel sens donner au \emph{calcul} d'une machine de Turing non
        déterminite ?
    \item Dans quel sens une machine de Turing est-elle proche d'un ordinateur
        classique ?
    \item En quoi les classes de complexités sont liées à l'efficacité des
        algorithmes ?
    \item Pourquoi étudier principalement la décidabilité plutôt que la
        calculabilité ?
\end{itemize}

\secreferences

\begin{itemize}
    \item \reference{Car}{Langages formels, calculabilité et complexité}{\bsc{Carton}}{à la BU/LSV}{Très bonne référence couvrant beaucoup de bases. Se méfier de certaines preuves faites un peu rapidement.}  

    \item \temporary{Sipser}
    \item \temporary{Arora Barack}
    \item \temporary{Autebert}
    \item \temporary{Papadimitriou}
\end{itemize}


\secdev

\begin{itemize}
    \item \temporary{Equivalence entre fonctions calculables et récursives}
    \item \temporary{Indécidabilité de l'arrêt et applications à quelques problèmes indécidables}
    \item \temporary{Equivalence entre deux variantes des machines de Turing}
    \item \temporary{Théorème de Cook}
    \item \temporary{Théorème de Savitch}
\end{itemize}


\end{document}
