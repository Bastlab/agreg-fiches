documentclass{agregfiche}

\title{Leçon 929 -- Lambda-calcul pur comme modèle de calcul. Exemples.}

\begin{document}

\maketitle

\secrapports

\begin{rapport}{2018}

Il s'agit de présenter un modèle de calcul : le lambda-calcul pur.
Il est important de faire le lien avec au moins un autre modèle de calcul, par exemple les machines de Turing ou les fonctions récursives.
Néanmoins, la leçon doit traiter des spécificités du lambda-calcul.
Ainsi, le candidat doit motiver l'intérêt du lambda-calcul pur sur les entiers et pourra aborder la façon dont il permet de définir et d'utiliser des types de données (booléens, couples, listes, arbres).

\end{rapport}

\secindispensables

\begin{itemize}
    \item Définitions : lambda-termes, alpha-équivalence, substitution.
    \item $\beta$-redex, $\beta$-réduction, confluence.
    \item Représentation des données : entiers, booléens.
    \item Fonctions représentables, fortement représentables.
    \item Exemples : fonctions arithmétiques, conditionnelle.
    \item Expressivité : équivalence avec les fonctions primitives.
\end{itemize}

\secpieges

\begin{itemize}
    \item Ne pas sortir du cadre "modèle de calcul".
\end{itemize}

\secidees

\begin{itemize}
    \item Types de données : couples, tuples, listes, arbres, et leurs primitives.
    \item Développements finis.
\end{itemize}

\secquestionsclassiques

\begin{itemize}
    \item Donner quelques applications non-triviales du tri ?
    \item Pourquoi veut-on un tri stable ? Un tri en place ? Un tri en ligne ?
    \item Coût des les opérations élémentaires sur les listes et les tableaux ?
    \item Comment améliorer le tri rapide pour qu'il soit optimal dans tous les cas ?
\end{itemize}

\secreferences

\begin{itemize}
    \item \input{Krivine}
    \item \temporary{Barendregt}
\end{itemize}


\secdev

\begin{itemize}
    \item \temporary{Equivalence entre fonctions récursives et fonctions représentables}
    \item \temporary{Le minimisation est fortement représentable}
    \item \temporary{Confluence du lambda-calcul}
\end{itemize}


\end{document}
