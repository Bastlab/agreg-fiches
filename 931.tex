\documentclass{agregfiche}

\title{Leçon 931 --  Schémas algorithmiques.  Exemples et applications.}

\begin{document}

\maketitle

\secrapports

\begin{rapport}{2018}

Cette leçon permet au candidat de présenter différents schémas algorithmiques, en particulier « diviser
pour régner », programmation dynamique et approche gloutonne. Le candidat pourra choisir de se
concentrer plus particulièrement sur un ou deux de ces paradigmes. Le jury attend du candidat qu’il
illustre sa leçon par des exemples variés, touchant des domaines différents et qu’il puisse discuter les
intérêts et limites respectifs des méthodes. Le jury ne manquera pas d’interroger plus particulièrement le
candidat sur la question de la correction des algorithmes proposés et sur la question de leur complexité,
en temps comme en espace.

\end{rapport}

\begin{rapport}{2017 - rapport concernant la 902, diviser pour regner}
Cette leçon permet au candidat de proposer différents algorithmes utilisant le paradigme diviser pour
régner. Le jury attend du candidat que ces exemples soient variés et touchent des domaines différents.
Un calcul de complexité ne peut se limiter au cas où la taille du problème est une puissance exacte de
2, ni à une application directe d’un théorème très général recopié approximativement d’un ouvrage de
la bibliothèque de l’agrégation.
\end{rapport}

\begin{rapport}{2017 - rapport concernant la 907, programmation dynamique}
Même s’il s’agit d’une leçon d’exemples et d’applications, le jury attend des candidats qu’ils présentent
les idées générales de la programmation dynamique et en particulier qu’ils aient compris le caractère
générique de la technique de mémoïsation. Le jury appréciera que les exemples choisis par le candidat
couvrent des domaines variés, et ne se limitent pas au calcul de la longueur de la plus grande sous-
séquence commune à deux chaînes de caractères.
Le jury ne manquera pas d’interroger plus particulièrement le candidat sur la question de la correction
des algorithmes proposés et sur la question de leur complexité en espace.
\end{rapport}


\secindispensables


\secidees

\secasavoir

\secpieges



\secquestionsclassiques


\secreferences


\secdev


\end{document}
