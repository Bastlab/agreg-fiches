\documentclass{article}
\usepackage[utf8]{inputenc}
\usepackage[french]{babel}

\usepackage{paralist}
\usepackage{amsmath,amssymb}
\usepackage{multicol}
\usepackage{url}

\title{Leçon 930 -- Sémantique des langages de programmation. Exemples.}

\newcommand{\indispensables}{\section{Les indispensables}}
\newcommand{\rapports}{\section{Extraits du Rapport}}
\newcommand{\pieges}{\section{Les pièges classiques}}
\newcommand{\idees}{\section{Idées}}
\newcommand{\references}{\section{Références}}
\newcommand{\dev}{\section{Dev}}
\newcommand{\questionsclassiques}{\section{Questions classiques}}

\begin{document}

\maketitle

\secrapports

\begin{rapport}{2017}

    L’objectif est de formaliser ce qu’est un programme : introduction des
    sémantiques opérationnelle et dénotationnelle, dans le but de pouvoir faire
    des preuves de programmes, des preuves d’équivalence, des preuves de
    correction de traduction. Ces notions sont typiquement introduites sur un
    langage de programmation (impératif) jouet. On peut tout à fait se limiter à
    un langage qui ne nécessite pas l’introduction des CPOs et des théorèmes de
    point fixe généraux. En revanche, on s’attend ici à ce que les liens entre
    sémantique opérationnelle et dénotationnelle soient étudiés (toujours dans le
    cas d’un langage jouet). Il est aussi important que la leçon présente des
    exemples d’utilisation des notions introduites, comme des preuves
    d’équivalence de programmes ou des preuves de correction de programmes.

\end{rapport}

\secindispensables

\begin{itemize}
    \item La sémantique opérationnelle (petit/grand pas) et dénotationnelle
        des expressions arithmétiques élémentaires.
    \item Les sémantiques des commandes du langage jouet choisi, en en proposant 
        aux moins deux différentes
    \item Démontrer des résultats d'adéquation et de correction entre les 
        sémantiques
    \item Donner des exemples de programmes, avec une interprétation de leur 
        correction dans une des sémantiques présentées
    \item Justifier le choix des constructeurs du langage utilisé
    \item Étudier les effets et leur traitement dans les différents modèles 
        (non-terminaison, non-déterminisme, etc)
    \item Faire le lien avec d'autres leçons~: 
        sémantique petit pas pour le lambda-calcul et les machines de Turing,
        sémantique à grand pas pour la logique (arbres de preuve),
        sémantique dénotationnelle pour les expressions rationnelles (langage).
    \item Utiliser la sémantique pour énoncer des propriétés (correction d'un
        programme, terminaison, validité d'une optimisation...)
\end{itemize}

\secpieges

\begin{itemize}
    \item Il n'est pas nécessaire d'aller vers les langages fonctionnels
        pour faire une bonne leçon, IMP suffit largement !
    \item Bien faire attention aux références qui prennent des directions 
        \emph{incompatibles} très tôt. Par exemple, si le domaine de définition
        des variables est $\mathbb{N}$ alors l'opération de soustraction 
        doit être \emph{binaire}, alors que dans le cas de $\mathbb{Z}$ on peut 
        se contenter d'une soustraciton \emph{unaire}.
    \item \temporary{Le grand classique des CPO plats, et de la boucle while qui converge 
        en un nombre fini d'étapes... Sauf que c'est pour une valeur fixée !}
    \item Il faut présenter les trois principales sémantiques (petit pas, grand
        pas, dénotationnel) mais ne pas oublier de \emph{justifier} leur intérêt
        propre ! Se demander pourquoi on en a trois est d'ailleurs pertinent...
\end{itemize}

\secidees

Il n'est pas avisé de suivre la recommandation du jury sur les CPO si 
on a les connaissances suffisantes, car utiliser des unions ensemblistes 
complexifie l'écriture, la lecture, et augmente la taille des preuves.
En revanche, on découpe soigneusement la
partie expression de la partie commandes.

Il \emph{faut} évoquer les différentes utilisations des différentes
sémantiques. Les opérationnelles sont pratiques pour implémenter, les
dénotationnelles pour raisonner. La notion de calcul permet-elle de définir
une complexité ? Peut-on décrire le comportement de programmes qui ne
terminent pas ? 

Attention, l'introduction de CPO peut amener à des questions triviales, mais
auxquelles il faut savoir répondre~: pourquoi le sup dans les fonctions est
le sup terme à terme ? Propriété de la chaine croissante ? etc ...

\secquestionsclassiques

\begin{itemize}
    \item Quel est l'intérêt des différentes sémantiques ?
    \item Pour quelle sémantique la notion de complexité est-elle 
        la plus naturelle ?
    \item Pour le programme d'une centrale nucléaire, quelle sémantique 
        paraît la plus adaptée ?
    \item Peut-on faire un lien entre la sémantique dénotationnelle
        et la preuve de programme ?
    \item Calculer la sémantique dénotationnelle d'un programme 
        avec une boucle while simple (division euclidienne, fibonacci...)
    \item Conaissez-vous un langage pour lequel il n'y a pas 
        de théorème d'adéquation ? De théorème de correction ?
    \item Connaissez-vous un autre moyen de définir une sémantique ?
        (Jeux, plus faible précondition...)
    \item \emph{(si les CPO ont été introduits)}
        Justifier que le $(\sup f_i) (x) = \sup f_i (x)$
        si $f_i$ est une famille dirigée d'applications 
        scott-continues.
    \item Comment ajouter un choix non déterministe dans 
        vos sémantiques ?
    \item Comment ajouter un choix probabiliste (pièce non biaisée par exemple)
        ?
\end{itemize}

\secreferences

\begin{itemize}
\item \temporary{Glynn Winskel}
\item \temporary{\url{http://www.cl.cam.ac.uk/~gw104/dens.pdf}}
    (aussi Glynn Winskel)
\end{itemize}

\secdev

\begin{itemize}
\item \temporary{Complétude de Hoare}
\item \temporary{Équivalence de sémantiques}
\end{itemize}


\end{document}
