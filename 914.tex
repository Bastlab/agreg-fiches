\documentclass{agregfiche}

\title{Leçon 914 -- Décidabilité et indécidabilité. Exemples.}

\begin{document}

\maketitle

\secrapports

\begin{rapport}{2018}
Le programme de l'option offre de très nombreuses possibilités d'exemples. Si les exemples classiques de problèmes sur les machines de Turing figurent naturellement dans la leçon, le jury apprécie des exemples issus d'autres parties du programme : théorie des langages, logique\dots
Le jury portera une attention particulière à une formalisation propre des réductions, qui sont parfois très approximatives.
\end{rapport}

\secindispensables

\begin{itemize}
\item Définitions des langages et problèmes décidables/récursivement énumérables.
\item Problème de l'arrêt.
\item Fonctions calculables, notion de réduction.
\end{itemize}

\secasavoir

\begin{itemize}
    \item Théorème de Rice.
	\item Problème de correspondance de Post.
	\item Plein d'exemple.
\end{itemize}


\secidees
% to complete
\begin{itemize}
\item Mettre en parallèle résultats positifs (décidabilité) et négatifs (indécidabilité).
\item Exemples choisis de décidabilité/indécidabilité en langages formels.
\item Exemples choisis de décidabilité/indécidabilité en logique. Liens avec l'incomplétude.
\item (dur) un peu de hiérarchie arithmétique.
\end{itemize}

\secpieges

\begin{itemize}
\item Etre très clair sur les définitions de départ et l'exigence de calculabilité dans les réductions.
\item Différence récursivement énumérable/décidable/indécidable.
\item Enoncer Rice sans erreur.
\item Attention à la représentation des données en entrée.
\end{itemize}



\secquestionsclassiques
\begin{itemize}
\item Justifier la propriété des réduction.
\item Expliquer ce que signifie Rice.
\item Décidable/indécidable dans l'arithmétique: Peano, Presburger, Skolem.
\item Décidable/indécidable pour les grammaires algébriques: vacuité, universalité, problème du mot et intersections, inclusion dans un rationnel\dots (Carton)
\item Machines linéairement bornées: vacuité, mot, universalité ? (Carton)
\end{itemize}

\secreferences
\begin{itemize}
\item \reference{Car}{Langages formels, calculabilité et complexité}{\bsc{Carton}}{à la BU/LSV}{Très bonne référence couvrant beaucoup de bases. Se méfier de certaines preuves faites un peu rapidement.}  

\end{itemize}

\secdev
%% sketchy, to complete
\begin{itemize}
\item \dev{Problème de l'arrêt et théorème de Rice.}{[Car]}{}{914}{Facile, donc on prêtera attention à la propreté de la rédaction.}

\item \dev{Problèmes indécidables pour les grammaires algébriques.}{[Car]}{}{914,923}{Ambigüité, universalité. Insister sur la réductio, pas sur la notion de grammaire.}

\item \dev{Décidabilité de l'arithmétique de Presburger}{[Car]}{Thm 3.63 p.164}{909,914,924}{Idée générale simple, mais attention aux détails.}

%\item \dev{Elimination des quantificateurs dans la théorie des ordres linéaires}{}{}{914,924}{}

\item \dev{Indécidabilité de l'arithmétique de 
\bsc{Peano}}{}{}{914,924}{Passer par l'encodage des fonctions 
calculables. Assez long si on fait tout.}


\end{itemize}


\end{document}
