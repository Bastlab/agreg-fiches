
\documentclass{agregfiche}

\title{Leçon 901 - Structures de données. Exemples et applications.}

\begin{document}
\maketitle

\secrapports
\begin{rapport}{2018}
	Le mot algorithme ne figure pas dans l’intitulé de cette leçon, même si l’utilisation des structures de
	données est évidemment fortement liée à des questions algorithmiques. La leçon doit donc être orientée
	plutôt sur la question du choix d’une structure de données. Le jury attend du candidat qu’il présente
	différents types abstraits de structures de données en donnant quelques exemples de leur usage avant de
	s’intéresser au choix de la structure concrète. Le candidat ne peut se limiter à des structures linéaires
	simples comme des tableaux ou des listes, mais doit présenter également quelques structures plus
	complexes, reposant par exemple sur des implantations à l’aide d’arbres. Les notions de complexité des
	opérations usuelles sur la structure de données sont bien sûr essentielles dans cette leçon.
\end{rapport}

\secindispensables

\begin{itemize}
	\item  Types abstraits, cas linéaire et non linéaire
	\item Interface abstraite et implan-
	tation concrète.
	\item Complexité des opérations usuelles
\end{itemize}

\secasavoir

\begin{itemize}
	\item  Tableaux, listes, piles, files, file de priorité, arbres, graphes (orientés et non orientés ensembles, dictionnaires
	\item Notion de coût amorti.
	\item Pour un même type de donnée (e.g un graphe), exemples de choix de représentations optimisés selon l'ensemble de données possibles (e.g graphe sparse ou dense)
	\item Influence de la structure de données sur un algorithme.
	\item Notion de types persistant et mutable.
	
\end{itemize}

\secidees

\begin{itemize}
	\item Structure avancé : Arbre rouge-noir , tas de Fibonnacci, Union-Find
	\item Structures paresseuses
	\item Représentation des entiers et float en machine.
\end{itemize}

\secpieges

\begin{itemize}
	\item Les algorithmes ne doivent pas être le coeur de la leçon, mais bien les structure de données.
	\item Les dessins sont souvent très appropriés pour représenter des structures de données et leurs évolutions.
	\item Bien avoir en tête les abstractions. Par exemple, un dictionnaire ne porte pas nécéssairement sur des strings.
	\item Attention à ne pas s'éparpiller avec trop de structures différentes. Il n'est pas forcément nécéssaire d'écrire tous les interfaces des types abstraits.
	\item La complexité, c'est bien, la comparaison des complexités, c'est mieux.
	\item Ne pas être trop lié à un langage de programmation.
	\item Les structures avancés introduites doivent être accompagné d'une application.
\end{itemize}

\secquestionsclassiques

\begin{itemize}
	\item Pourquoi avoir une approche par type abstraits ?
	\item Quels structure de donnés sont en pratique les plus utilisés, et intégrés nativement dans les languages de programmations ?
	\item Connaissez vous l'implémentation des [insérerer ici une structure] en pratique utilisé ?
	\item Avec les listes chainé/doublement chainé, quelles opérations sont plus ou moins complexe ?
	\item Le choix du type, par exemple persistent et mutable, est souvent lié à des soucis d'efficacité. Y-a-t'il d'autre préoccupations lors de la programmation influencées par ce choix ?
\end{itemize}

\secreferences

\begin{itemize}
\item \reference{Cor}{Algorithmique}{Cormen}{à la BU/LSV}{La bible de l'algorithmique, avec toutes les bases. Attention, les calculs avec des probas sont parfois faux.}  

\item \reference{Bea}{Éléments d'algorithmique}{D. \bsc{Beauquier}, J. \bsc{Berstel}, Ph. \bsc{Chrétienne}}{à la BU/LSV}{Bonne référence pour l'algo, pleins de dessins et de preuves. Un peu vieillissant et devenu rare.}  

\end{itemize}

\secdev

\begin{itemize}
\item \dev{Complexité et correction du tri par tas}{[Cor]}{3rd edition, p.154}{901,903}{Simple dans le fond, mais à bien faire formellement, et pédagogiquement. Dessins et exemples bienvenue.}

\item \dev{Hachage Parfait}{[Cor]}{p. 258}{901,921}{}

\item \dev{Arbre Binaire de Recherche optimaux}{Corf}{p. 397}{901,921}{}

\end{itemize}


\end{document}