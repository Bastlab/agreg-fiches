
\documentclass{agregfiche}

\title{Leçon 923 - Analyses lexicale et syntaxique. Applications.}

\begin{document}
\maketitle

\secrapports
\begin{rapport}{2018}
    Cette leçon ne doit pas être confondue avec la 909, qui s’intéresse aux seuls langages rationnels, ni avec
    la 907, sur l’algorithmique du texte.
    Si les notions d’automates finis et de langages rationnels et de grammaires algébriques sont au cœur
    de cette leçon, l’accent doit être mis sur leur utilisation comme outils pour les analyses lexicale et
    syntaxique. Il s’agit donc d’insister sur la différence entre langages rationnels et algébriques, sans perdre
    de vue l’aspect applicatif : on pensera bien sûr à la compilation. Le programme permet également des
    développements pour cette leçon avec une ouverture sur des aspects élémentaires d’analyse sémantique.
\end{rapport}

\begin{rapport}{2017}
[\dots]    On pourra s’intéresser à la transition entre analyse lexicale et analyse syntaxique, et on pourra présenter les outils associés classiques, sur un exemple
    simple. Les notions d’ambiguïté et l’aspect algorithmique doivent être développés. La présentation
    d’un type particulier de grammaire algébrique pour laquelle on sait décrire un algorithme d’analyse
    syntaxique efficace sera ainsi appréciée. [\dots]
    \end{rapport}

\secindispensables

\begin{itemize}
	\item Grammaires et Langages algébriques. Existence de langages non algébriques. Propriétés de clôture
    des langages algébriques.
    \item Chaîne de compilation. Analyse lexicale. Analyse syntaxique (principes de l’analyse descendante
    et ascendante). 
\end{itemize}

\secasavoir

\begin{itemize}
    \item Formes normales de Chomsky
	\item Automate à pile et équivalence avec les langages algébriques.
    \item Analyse sémantique élémentaire (arbre de syntaxe abstraite, table des symboles,
    analyse de portée, typage, ...).
    \item Grammaires LL(k) ou LR(k)
\end{itemize}

\secidees

\begin{itemize}
	\item 
\end{itemize}

\secpieges

\begin{itemize}
    \item Des DESSINS, de la chaîne de compilation, d'arbre de dérivation, d'automate, de règles de grammaires,...
    \item Mettre des exemples. Bien s'entraîner à faire tourner les algorithmes sur des exemples (notamment pour l'analyse lexical)
	\item Attention à ne pas trop se perdre sur certain théorique. On peut notamment se perdre et ne pas savoir répondre aux questions autour des parseurs LL, LR, LALR et SLR.

\end{itemize}

\secquestionsclassiques

\begin{itemize}
	\item Qu'utilisent en pratique les programmeurs pour écrire des parseurs ? Quelles types de grammaire est reconnu par ces outils ?
% lex and yacc, LALR
\item Peut-on comparer les parseurs LALR, LR, LL, SLR ? Si  des inclusions sont strictes, avez-vous des exemples ?
% LALR < LR < SLR, LALR(j) incomparable avec LL(k)
\item Quelles sont les applications de l'analyse lexicale et syntaxique ? Avez-vous un exemple où la complexité est critique ?
\end{itemize}

\secreferences

\begin{itemize}
\item \item My dummy ref\\
Elle est vraiment chouette

\end{itemize}

\secdev

\begin{itemize}
\item \item My dummy dev\\
Elle est vraiment chouette

\end{itemize}


\end{document}