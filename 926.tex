
\documentclass{agregfiche}

\title{Leçon 926 - Analyse des algorithmes : complexité. Exemples.}

\begin{document}
\maketitle

\secrapports
\begin{rapport}{2018}
    Il s’agit ici d’une leçon d’exemples. Le candidat prendra soin de proposer l’analyse d’algorithmes portant
    sur des domaines variés, avec des méthodes d’analyse également variées : approche combinatoire ou
    probabiliste, analyse en moyenne ou dans le pire cas.
    Si la complexité en temps est centrale dans la leçon, la complexité en espace ne doit pas être négligée.
    La notion de complexité amortie a également toute sa place dans cette leçon, sur un exemple bien
    choisi, comme union find (ce n’est qu’un exemple).
\end{rapport}

\secindispensables

\begin{itemize}
	\item Méthodes d'analyse
    \item Complexité en temps (meilleur cas, pire cas, moyenne) et en espace.
    \item Des exemples.
\end{itemize}

\secasavoir

\begin{itemize}
	\item Analyse des algorithmes : relations de comparaison $O$, $\theta$ et $\Omega$.
    \item Exemple d’analyse en moyenne : recherche d’un élément dans un tableau.
    \item Des algorithmes et leurs complexité dans différent domaine (minimisation d'automate, plus court chemins, unification, recherche... )
    	\item Complexité amortie par méthode de l'agrégat.
\end{itemize}

\secidees

\begin{itemize}
    	\item Complexité amortie par méthode comptable ou potentiel.
        \item FFT
\end{itemize}

\secpieges

\begin{itemize}
	\item Connaître la complexité des diverses opérations de bases de chacun des algorithmes présenté.
    \item Il faut trouver un équilibre entre la présentation d'exemples, et la présentation formelle des notions théoriques.
    \item Si certains points sont présenté de manière informelle, ne pas les appeler proposition ou théorème.
    \item Faire le lien avec certaines méthodes et certains paradigme de programmations.
\end{itemize}

\secquestionsclassiques

\begin{itemize}
	\item Lors d'une analyse de complexité en moyenne, que supposes-t-on sur la distribution des entrées ? Est-ce pertinent ?
    \item Pourquoi se contente-t-on de $O$, plutôt que de calcul exacts ?
    \item Pouvez vous dérouler tel algorithme sur tel exemple ?
    \item Questions de détails d'implémentation/de complexité des opérations élémentaire sur un algorithme présenté.
    \item Dans tel cas pratique, quelle algorithme vaut-il mieux utiliser ?
    \item Savez-vous si tel algorithme est optimal ?
    
\end{itemize}

\secreferences

\begin{itemize}
\item \reference{Cor}{Algorithmique}{Cormen}{à la BU/LSV}{La bible de l'algorithmique, avec toutes les bases. Attention, les calculs avec des probas sont parfois faux.}  

\item \reference{Bea}{Éléments d'algorithmique}{D. \bsc{Beauquier}, J. \bsc{Berstel}, Ph. \bsc{Chrétienne}}{à la BU/LSV}{Bonne référence pour l'algo, pleins de dessins et de preuves. Un peu vieillissant et devenu rare.}  


\end{itemize}

\secdev

\begin{itemize}
\item \dev{ Correction totale et/ou complexité de \bsc{Dijkstra}}{Cor, Beau}{p. ?}{925,926,927}{Long de faire correction et complexité, doit-être adapté selon la leçon.}

\item \dev{Complexité du tri rapide}{[Cor,Bea]}{}{903,931}{Bien faire attention au proba du Cormen, aller voir Beauquier est pertinent. Avoir une idée de l'écart type des performances.}

\end{itemize}


\end{document}