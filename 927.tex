
\documentclass{agregfiche}

\title{Leçon 927 - Exemples de preuve d’algorithme :  correction, terminaison.}

\begin{document}
\maketitle

\secrapports
\begin{rapport}{2018}
    Le jury attend du candidat qu’il traite des exemples d’algorithmes récursifs et des exemples d’algorithmes itératifs.
    En particulier, le candidat doit présenter des exemples mettant en évidence l’intérêt de la notion
    d’invariant pour la correction partielle et celle de variant pour la terminaison des segments itératifs.
    Une formalisation comme la logique de \bsc{Hoare} pourra utilement être introduite dans cette leçon, à
    condition toutefois que le candidat en maîtrise le langage. Des exemples non triviaux de correction
    d’algorithmes seront proposés. Un exemple de raisonnement type pour prouver la correction des algorithmes gloutons pourra éventuellement faire l’objet d’un développement.
\end{rapport}

\secindispensables

\begin{itemize}
	\item Notions de correction et terminaison
    \item Méthodes de preuves
    \item Des EXEMPLES de preuves
\end{itemize}

\secasavoir

\begin{itemize}
	\item Assertions, préconditions et
    postconditions, invariants et variants de boucles, logique de \bsc{Hoare}, induction structurelle.
\end{itemize}

\secidees

\begin{itemize}
	\item Correction et complétude de la logique de Hoare
    \item Algorithmes parallèles
\end{itemize}

\secpieges

\begin{itemize}
	\item Il peut être intéressant de faire des liens avec des notions mathématique bien connu (\bsc{Syracuse}, Conjecture de \bsc{Collatz})
    \item Leçons très facile à motiver, alors faite le !
\end{itemize}

\secquestionsclassiques

\begin{itemize}
	\item La terminaison est-elle décidable ? La correction ?
    \item Peut-on automatiser ce genre d'analyse ? Quelles outils connaissez-vous ?
    \item Comment influe les paradigmes de programmation ?
    \item  Existe-t-il une variante de la logique de \bsc{Hoare} pour les programmes avec des structures à champs modifiable et des pointeurs vers ces structures ?
    \item Quelle est la valeur d'une preuve de programme dans le monde réel ?
    \item Si on ne cherche pas une preuve parfaite mais uniquement des garanties partielle, quelles techniques peut-on utiliser ?
     
\end{itemize}

\secreferences

\begin{itemize}
\item \reference{Cor}{Algorithmique}{Cormen}{à la BU/LSV}{La bible de l'algorithmique, avec toutes les bases. Attention, les calculs avec des probas sont parfois faux.}  

\item \reference{Win}{The formal semantics of programming langages}{\bsc{Winskel}}{à la BU/LSV}{}
\end{itemize}

\secdev

\begin{itemize}
\item \dev{Preuve de la factorielle en \bsc{Hoare}}{[Win]}{ch6.6 p.93}{927}{Peut être l'occasion de parler des problèmes d'automatisation.}

\item \dev{Preuve de correction d'un algorithme}{[Cor,Beau]}{}{927,?}{Dijkstra, KMP, unification}

\end{itemize}


\end{document}