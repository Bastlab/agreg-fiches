\documentclass{agregfiche}

\title{Leçon 912 - Fonctions récursives primitives et non primitives. Exemples}

\begin{document}
\maketitle

\secrapports
\begin{rapport}{2018}
	Il s’agit de présenter un modèle de calcul : les fonctions récursives. S’il est bien sûr important de faire
	le lien avec d’autres modèles de calcul, par exemple les machines de Turing, la leçon doit traiter des
	spécificités de l’approche. Le candidat doit motiver l’intérêt de ces classes de fonctions sur les entiers et
	pourra aborder la hiérarchie des fonctions récursives primitives. Enfin, la variété des exemples proposés
	sera appréciée.
\end{rapport}

\secindispensables

\begin{itemize}
	\item Des \textbf{exemples}, partout, pour tout. (somme fini, produit fini...) 
	\item Définitions des fonctions primitives récursives de bases, primitives récursives et non-primitive récursive ($\mu$-récursives, i.e avec minimisation non borné).
	\item Prédicat primitifs et fonctions caractéristique. Minimisation borné.
  
\end{itemize}

\secasavoir

\begin{itemize}
	\item \bsc{Ackerman}
	\item Lien avec les TM, ensembles récursifs, récursivement énumérable et fonctions partiels.
\end{itemize}


\secidees

\begin{itemize}

	\item Hiérarchie de \bsc{Grzegorczyk}
	\item Lien avec le $\lambda$-calcul
	\item Théorème d'itération ($s_{mn}$),
\end{itemize}

\secpieges

\begin{itemize}
	\item Mettre des exemples intéressants.
    \item Faire des liens avec les constructions des langages de programmation.
	\item Ne pas définir les fonctions récursives primitives autrement que à partir de celles de bases.
\end{itemize}

\secquestionsclassiques

\begin{itemize}
	\item Pourquoi considérer les fonctions récursive primitives ?
	\item Différence entre la récursion définit ici et celle des langages de programmation.
    \item Quelle construction capture la boucle FOR ? La boucle WHILE ?
	\item L'encodage d'un problème est-il toujours calculable ? Pourquoi sont définis ainsi les fonctions récursives primitives de bases ?
	\item Exemple d'argument diagonal.
	\item Montrer que telle fonction est primitive récursive.
	
\end{itemize}
\secreferences

\begin{itemize}
\item \reference{Wol}{, Introduction à la calculabilité : cours et exercices corrigés}{\bsc{Wolper}}{à la BU/LSV}{Appréciable pour sa pédagogie, et la compréhension des concepts majeurs.}  

\item \reference{Car}{Langages formels, calculabilité et complexité}{\bsc{Carton}}{à la BU/LSV}{Très bonne référence couvrant beaucoup de bases. Se méfier de certaines preuves faites un peu rapidement.}  

\end{itemize}

\secdev

\begin{itemize}
    \item  \dev{X n'est pas récursive primitive}{[Cori],[Car]}{}{912}{X pour Ackerman, avec une preuve un peu compliqué dans le Cori, où X = isPrime dans le Carton.}

    \item \dev{Une fonction \bsc{Turing} calculable est $\mu$-recursive.}{[Wol]}{}{912,913}{Preuve précise mais non pédagogique dans le Cori, claire mais non précise dans le Wolper...}

\end{itemize}


\end{document}