
\documentclass{agregfiche}

\title{Leçon 932 - Fondements des bases de données relationnelles}

\begin{document}
\maketitle

\secrapports
\begin{rapport}{2018}
    Le cœur de cette nouvelle leçon concerne les fondements 
    théoriques des bases de données relationnelles :
    présentation du modèle relationnel, approches logique et 
    algébrique des langages de requêtes, liens entre
    ces deux approches.
    Le candidat pourra ensuite orienter la leçon et proposer des 
    développements dans des directions diverses : complexité de 
    l’évaluation des requêtes, expressivité 
    des langages de requête, requêtes récursives, contraintes 
    d’intégrité, aspects concernant la conception 
    et l’implémentation, optimisation de
    requêtes...
\end{rapport}

\secindispensables

\begin{itemize}
	\item  modèle relationnel, algèbre relationnelle, calcul 
	relationnel.
    \item théorème de \bsc{Codd}.
\end{itemize}

\secasavoir

\begin{itemize}
	\item Calcul conjonctif.
    \item Indécidabilité de la satisfiabilité (\bsc{Trakhtenbrot}). 
    De l'indépendance de domaine.
\end{itemize}

\secidees

\begin{itemize}
    \item Minimisation.
    \item Expressivité, limites et extensions.
	\item Dépendances fonctionnelles et contraintes d'intégrité. 
	Système d'\bsc{Armstrong}.
    \item Implémentations (B-arbres).
    \item Complexité.
\end{itemize}

\secpieges

\begin{itemize}
	\item Il ne suffit pas d'écrire les définitions. Il faut 
	comprendre leur liens avec les objets manipulé en pratique. 
    \item Donner des exemples de BDD, de requêtes, de résultats.
    \item Pour arriver au théorème de \bsc{Codd}, il faut beaucoup de 
    définitions, essayez d'alléger le tout avec des exemples, des 
    remarques, des dessins.
\end{itemize}

\secquestionsclassiques

\begin{itemize}
    \item Quelle est l'intérêt des bases de données relationnelles ?
	\item Donner une exemple de requête non domaine indépendante. 
	Donner deux BDD qui l'illustre.
    \item Est-ce que dans le modèle relationnel on peut avoir des 
    duplicats ? Dans SQL ?
    \item Connaissez vous le principe des clés primaires ?
    \item Pouvez vous écrire la requête permettant de ...?
    \item Quelle est le résultat de la construction du théorème de 
    \bsc{Codd} sur tel exemple ?
    \item D'où vient l'indécidabilité de la satisfiabilité ?
    \item SQL est-il plus expressif ? Si oui, donner des exemples de 
    constructions supplémentaire.
    \item Quelle est l'utilité des contraintes fonctionnelles ?
\end{itemize}

\secreferences

\begin{itemize}
\item \item My dummy ref\\
Elle est vraiment chouette

\end{itemize}

\secdev

\begin{itemize}
\item \item My dummy dev\\
Elle est vraiment chouette

\end{itemize}


\end{document}